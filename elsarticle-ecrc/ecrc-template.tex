
% Template for Elsevier CRC journal article
% version 1.2 dated 09 May 2011

% This file (c) 2009-2011 Elsevier Ltd.  Modifications may be freely made,
% provided the edited file is saved under a different name

% This file contains modifications for Procedia Computer Science
% but may easily be adapted to other journals

% Changes since version 1.1
% - added "procedia" option compliant with ecrc.sty version 1.2a
%   (makes the layout approximately the same as the Word CRC template)
% - added example for generating copyright line in abstract

%-----------------------------------------------------------------------------------

%% This template uses the elsarticle.cls document class and the extension package ecrc.sty
%% For full documentation on usage of elsarticle.cls, consult the documentation "elsdoc.pdf"
%% Further resources available at http://www.elsevier.com/latex

%-----------------------------------------------------------------------------------

%%%%%%%%%%%%%%%%%%%%%%%%%%%%%%%%%%%%%%%%%%%%%%%%%%%%%%%%%%%%%%
%%%%%%%%%%%%%%%%%%%%%%%%%%%%%%%%%%%%%%%%%%%%%%%%%%%%%%%%%%%%%%
%%                                                          %%
%% Important note on usage                                  %%
%% -----------------------                                  %%
%% This file should normally be compiled with PDFLaTeX      %%
%% Using standard LaTeX should work but may produce clashes %%
%%                                                          %%
%%%%%%%%%%%%%%%%%%%%%%%%%%%%%%%%%%%%%%%%%%%%%%%%%%%%%%%%%%%%%%
%%%%%%%%%%%%%%%%%%%%%%%%%%%%%%%%%%%%%%%%%%%%%%%%%%%%%%%%%%%%%%

%% The '3p' and 'times' class options of elsarticle are used for Elsevier CRC
%% Add the 'procedia' option to approximate to the Word template
%\documentclass[3p,times,procedia]{elsarticle}
\documentclass[3p,times]{elsarticle}

%% The `ecrc' package must be called to make the CRC functionality available
\usepackage{ecrc}

%% The ecrc package defines commands needed for running heads and logos.
%% For running heads, you can set the journal name, the volume, the starting page and the authors

%% set the volume if you know. Otherwise `00'
\volume{00}

%% set the starting page if not 1
\firstpage{1}

%% Give the name of the journal
\journalname{Mathematical Biosciences}

%% Give the author list to appear in the running head
%% Example \runauth{C.V. Radhakrishnan et al.}
\runauth{}

%% The choice of journal logo is determined by the \jid and \jnltitlelogo commands.
%% A user-supplied logo with the name <\jid>logo.pdf will be inserted if present.
%% e.g. if \jid{yspmi} the system will look for a file yspmilogo.pdf
%% Otherwise the content of \jnltitlelogo will be set between horizontal lines as a default logo

%% Give the abbreviation of the Journal.  Contact the journal editorial office if in any doubt
\jid{procs}

%% Give a short journal name for the dummy logo (if needed)
\jnltitlelogo{Mathematical Biosciences}

%% Provide the copyright line to appear in the abstract
%% Usage:
%   \CopyrightLine[<text-before-year>]{<year>}{<restt-of-the-copyright-text>}
%   \CopyrightLine[Crown copyright]{2011}{Published by Elsevier Ltd.}
%   \CopyrightLine{2011}{Elsevier Ltd. All rights reserved}
\CopyrightLine{2011}{Published by Elsevier Ltd.}

%% Hereafter the template follows `elsarticle'.
%% For more details see the existing template files elsarticle-template-harv.tex and elsarticle-template-num.tex.

%% Elsevier CRC generally uses a numbered reference style
%% For this, the conventions of elsarticle-template-num.tex should be followed (included below)
%% If using BibTeX, use the style file elsarticle-num.bst

%% End of ecrc-specific commands
%%%%%%%%%%%%%%%%%%%%%%%%%%%%%%%%%%%%%%%%%%%%%%%%%%%%%%%%%%%%%%%%%%%%%%%%%%

%% The amssymb package provides various useful mathematical symbols
\usepackage{amssymb}
\usepackage{hyperref}
%% The amsthm package provides extended theorem environments
%% \usepackage{amsthm}

%% The lineno packages adds line numbers. Start line numbering with
%% \begin{linenumbers}, end it with \end{linenumbers}. Or switch it on
%% for the whole article with \linenumbers after \end{frontmatter}.
%% \usepackage{lineno}

%% natbib.sty is loaded by default. However, natbib options can be
%% provided with \biboptions{...} command. Following options are
%% valid:

%%   round  -  round parentheses are used (default)
%%   square -  square brackets are used   [option]
%%   curly  -  curly braces are used      {option}
%%   angle  -  angle brackets are used    <option>
%%   semicolon  -  multiple citations separated by semi-colon
%%   colon  - same as semicolon, an earlier confusion
%%   comma  -  separated by comma
%%   numbers-  selects numerical citations
%%   super  -  numerical citations as superscripts
%%   sort   -  sorts multiple citations according to order in ref. list
%%   sort&compress   -  like sort, but also compresses numerical citations
%%   compress - compresses without sorting
%%
%% \biboptions{comma,round}

% \biboptions{}

% if you have landscape tables
\usepackage[figuresright]{rotating}

% put your own definitions here:
%   \newcommand{\cZ}{\cal{Z}}
%   \newtheorem{def}{Definition}[section]
%   ...

% add words to TeX's hyphenation exception list
%\hyphenation{author another created financial paper re-commend-ed Post-Script}

% declarations for front matter

\begin{document}

\begin{frontmatter}

%% Title, authors and addresses

%% use the tnoteref command within \title for footnotes;
%% use the tnotetext command for the associated footnote;
%% use the fnref command within \author or \address for footnotes;
%% use the fntext command for the associated footnote;
%% use the corref command within \author for corresponding author footnotes;
%% use the cortext command for the associated footnote;
%% use the ead command for the email address,
%% and the form \ead[url] for the home page:
%%
%% \title{Title\tnoteref{label1}}
%% \tnotetext[label1]{}
%% \author{Name\corref{cor1}\fnref{label2}}
%% \ead{email address}
%% \ead[url]{home page}
%% \fntext[label2]{}
%% \cortext[cor1]{}
%% \address{Address\fnref{label3}}
%% \fntext[label3]{}

\dochead{}
%% Use \dochead if there is an article header, e.g. \dochead{Short communication}
%% \dochead can also be used to include a conference title, if directed by the editors
%% e.g. \dochead{17th International Conference on Dynamical Processes in Excited States of Solids}

\title{Utilization of environmental and epidemiological indicators in the study of malaria dynamics}

%% use optional labels to link authors explicitly to addresses:
%% \author[label1,label2]{<author name>}
%% \address[label1]{<address>}
%% \address[label2]{<address>}

\author{Raphael Felberg Levy}
\ead{raphael.levy147@gmail.com}

\author{Flavio Codeço Coelho}
\ead{flavio.codeco.coelho@fgv.br}

%\address{}

\address{Praia de Botafogo 190, Rio de Janeiro, Brazil}

\begin{abstract}
%% Text of abstract
This paper aims to analyze the behavior of malaria transmission in the Amazon region based on climatic and environmental changes, such as temperature, precipitation and deforestation, through proposed 
      modifications to the SIR and SEI models, in order to contribute to the study of applications
      of external effects on the evolution of the disease. 
The Trajetórias Project, developed by the Synthesis Center 
      on Biodiversity and Ecosystem Services (SinBiose/CNPq) was used as an initial reference for the study. This work employs a modified SIR/SEI methodology, based on work from Parham and Michael (2010) which takes into account rainfall and temperature, with further modifications to avoid delay equations. Primary results show that the model is too sensible on some parameters, and using values indicated by other papers did not give the same results, opening up future work to compare the modified model equations with the originals. With the obtained results, it was possible to verify a strong effect caused by increased contact of host and vectors on the transmission of the disease. 
%Malaria is an infectious disease transmitted by mosquitoes infected by protozoa of the genus \textit{Plasmodium}, with the Amazon region being considered an endemic area 
%      for the disease. This work aims to analyze the behavior of this transmission based on climatic and environmental changes, such as temperature, precipitation and deforestation, through proposed 
%      modifications to the SIR and SEI models, in order to contribute to the study of applications
%      of external effects on the evolution of the disease. 
     % The Trajetórias Project, developed by the Synthesis Center 
      %on Biodiversity and Ecosystem Services (SinBiose/CNPq) will be used as a reference base for the analyses.
\end{abstract}

\begin{keyword}
%% keywords here, in the form: keyword \sep keyword
Biological modelling \sep Malaria \sep Amazon \sep SIR \sep SEI

%% PACS codes here, in the form: \PACS code \sep code

%% MSC codes here, in the form: \MSC code \sep code
%% or \MSC[2008] code \sep code (2000 is the default)

\end{keyword}

\end{frontmatter}

%%
%% Start line numbering here if you want
%%
% \linenumbers

%% main text
\section{Introduction}
\label{}
The Amazon is one of the largest and most biodiverse tropical forests 
in the world, harboring numerous species of plants, animals, and 
microorganisms, including vectors and pathogens responsible for the 
transmission of various diseases. Among them, one of the most common 
is malaria, caused by protozoa of the genus \textit{Plasmodium}, 
transmitted by the bite of the infected female mosquito of the genus 
\textit{Anopheles}. It is present in 22 American countries, but the 
areas with the highest risk of infection are located in the Amazon 
region, encompassing nine countries, which accounted for $68\%$ of 
infection cases in 
2011 \cite{pimenta_orfano_bahia_duarte_rios-velasquez_melo_pessoa_oliveira_campos_villegas_etal_2015}. 
Although malaria is prevalent in the Americas, it is 
not limited to this continent and is found in countries in Africa and Asia, 
resulting in more than two million cases of infection and 445,000 deaths 
worldwide in 2016 \cite{doi:10.1146/annurev-micro-090817-062712}.

Notably, vector-borne disease transmission is closely related to 
environmental changes that interfere with the ecosystem of both 
transmitting organisms and affected organisms. In the case of the 
Amazon, agricultural and livestock settlements are among the factors
that most favor disease transmission, both due to the deforestation 
they cause for establishment and the clustering of people in environments 
close to the vector's habitat \cite{silva-nunes_malaria_amazon_2008}, 
especially by clustering non-immune migrants near these natural and 
artificial breeding sites \cite{DASILVANUNES2012281}.

Additionally, other factors such as rainfall, wildfires, and mining 
also significantly influence disease transmission in the region. These 
events result in habitat loss, ecosystem fragmentation, and climate 
changes, affecting the distribution and abundance of vectors and hosts, 
as well as their interaction with pathogens. Furthermore, population growth 
and urbanization also play a crucial role in disease spread, increasing 
human exposure to vectors and infection risks.

In this context, this work aims to investigate vector-borne disease 
transmission in the Amazon and analyze how environmental impacts 
influence the dynamics of malaria transmission, the ecological and 
socioeconomic factors affecting this spread, and possible prevention 
and control strategies. The main reference for this research is the 
Trajetórias Project, developed by the Center for Biodiversity and 
Ecosystem Services (SinBiose/CNPq), which is a dataset including 
environmental, epidemiological, economic, and socioeconomic indicators 
for all municipalities in the Legal Amazon, analyzing the spatial and 
temporal relationship between economic trajectories linked to the dynamics 
of agrarian systems, whether they are family-based rural or large-scale 
agricultural and livestock production, the availability of natural resources, 
and the risk of diseases \cite{Rorato2023}.

%% The Appendices part is started with the command \appendix;
%% appendix sections are then done as normal sections
%% \appendix

%% \section{}
%% \label{}

%% References
%%
%% Following citation commands can be used in the body text:
%% Usage of \cite is as follows:
%%   \cite{key}         ==>>  [#]
%%   \cite[chap. 2]{key} ==>> [#, chap. 2]
%%

%% References with BibTeX database:

\bibliographystyle{elsarticle-num}
\bibliography{<your-bib-database>}

%% Authors are advised to use a BibTeX database file for their reference list.
%% The provided style file elsarticle-num.bst formats references in the required Procedia style

%% For references without a BibTeX database:

% \begin{thebibliography}{00}
 \begin{thebibliography}{00}

%% \bibitem must have the following form:
%%   \bibitem{key}...
%%

\bibitem{pimenta_orfano_bahia_duarte_rios-velasquez_melo_pessoa_oliveira_campos_villegas_etal_2015} P. F. P. Pimenta et al., An overview of malaria transmission from the perspective of Amazon \emph{Anopheles vectors}, Memórias do Instituto Oswaldo Cruz, Vol. 110 (1) (2015), 23-47, \href{https://doi.org/10.1590/0074-02760140266}{https://doi.org/10.1590/0074-02760140266}.

% \end{thebibliography}
\end{thebibliography}

\end{document}

%%
%% End of file `ecrc-template.tex'. 